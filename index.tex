\documentclass[12]{article}
\usepackage{html}
\pagestyle{empty}

\begin{document}
\begin{center}
\Large
Computational Mechanics Brown-Bag Seminar
\end{center}

\begin{itemize}
\item[Time:] Wednesdays, 12:00-1:00

\item[Place:] CHL 244 (CHL Classroom)

\item[Purpose:] The computational mechanics brown-bag seminar is an
  opportunity to present and discuss topics related to the computer
  simulation of fluid and solid mechanics. Talks should be informal
  and from 20 to 50 minutes in length. Questions/polite interruptions
  are encouraged. Topics may include, but are not limited to,
  technical issues from ongoing research, completed work, tutorials on
  advanced computational methods, or presentations of classical papers
  that have been cited at least 100 times.  At irregular intervals,
  prizes will be awarded for least colorful presentation, best \LaTeX
  slide, and poorest handwriting.

\item[Mailing List:] To subscribe to our email list,
  \htmladdnormallink{click here}{http://lst.usace.army.mil/majordomosubunsub.htm}
  and select ls-cms.

\item[Information:] If you would like to give a presentation, please send a title, abstract, and date to Collin.A.Vaughan@usace.army.mil.


\end{itemize}


\begin{center}
\Large
Upcoming talks
\end{center}

\begin{center}
\Large
Year 2011
\end{center}

\begin{enumerate}

\item[Jun 15]Lessons learned from hurricane Katrina and IPET\\Bruce Ebersole (CHL)\\


\item[Sept 14]Stabilized Approximation to Degenerate Transport Problems via Filtering\\Lea Jenkins (Clemson University)\\
We analyze a stabilization technique for advection-dominated
flow problems.  Of particular interest are coupled parabolic/hyperbolic
problems, when the diffusion coefficient is zero in part of the domain. 
For certain discretization methods, the unstabilized, computed
approximations of these problems are highly oscillatory, and several
techniques have been proposed and analyzed to mitigate the effects of the
subgrid errors that contribute to the oscillatory behavior.  In this
paper, we modify a time-relaxation algorithm proposed by Adams and Stolz
in 2002 and further studied in Ervin, Layton, and Neda in 2007.  Our
modification introduces the relaxation operator as a post-processing step.
The operator is not time-dependent, so the discrete (relaxation) system
need only be factored once.  We discuss convergence results for the
algorithm and present numerical results for several model problems.

\end{enumerate}

\begin{center}
\Large
Prior talks
\end{center}

\begin{center}
\Large
Year 2011
\end{center}

\begin{enumerate}

\item[Apr 27]Computational Geometry Algorithms Library (CGAL)\\GoogleTechTalk by Andreas Fabri (Geometry Factory) and Sylvain Pion (INRIA)\\
\\
From www.cgal.org: The Computational Geometry Algorithms Library (CGAL), offers data structures and algorithms like triangulations(2D constrained triangulations and Delaunay triangulations in 2D and 3D, periodic triangulations in 3D), Voronoi diagrams (for 2D and 3D points, 2D additively weighted Voronoi diagrams, and segment Voronoi diagrams),polygons (Boolean operations, offsets, straight skeleton), polyhedra (Boolean operations), arrangements of curves and their applications (2D and 3D envelopes, Minkowski sums), mesh generation (2D Delaunay mesh generation and 3D surface and volume mesh generation, skin surfaces), geometry processing (surface mesh simplification, subdivision and parameterization, as well as estimation of local differential properties, and approximation of ridges and umbilics), alpha shapes, convex hull algorithms (in 2D, 3D and dD), search structures (kd trees for nearest neighbor search, and range and segment trees), interpolation (natural neighbor interpolation and placement of streamlines), shape analysis, fitting, and distances (smallest enclosing sphere of points or spheres, smallest enclosing ellipsoid of points, principal component analysis), and kinetic data structures.

\item[Apr 13]Mpi4py and Petsc4py: Using Python to develop Scalable PDE Solvers\\Lisandro Dalcin\\Centro Int. de Metodos Computacionales en Ingeniera\\(Presented by Chris Kees)\\
\\
PETSc is a suite of routines and data structures for the
scalable (parallel, MPI-based) solution of scientific applications
modeled by PDEs. PETSc for Python (aka petsc4py)
is a Cython-based wrapper to PETSc components like distributed
vectors and matrices, Krylov-based linear solvers,
Newton-based nonlinear solvers, and basic timesteppers.
petsc4py facilitates the Cython/SWIG/f2py wrapping of
C/C++/Fortran codes using PETSc. New mechanisms are
being worked on to cross language boundaries and easily
employ Python as an extension language.

\item[Apr 6]Matlab Routines for 2D Mesh Generation and Mesh Quality Evaluation\\Chris Massey (CHL)\\

\item[Mar 30]Applications of Peskin's Immersed Boundary Method\\ John Chrispell (Tulane University) \\
Many fluids are non-Newtonian and exhibit viscoelastic responses.
Examples include: food products like egg whites, and molten chocolate,
biological
fluids such as mucus and blood, as well as industrial materials like
paint, adhesives, and polymers. Here we discuss recent results obtained
using an immersed boundary framework to study the interaction between
immersed elastic structures and surrounding viscoelastic fluids.

\item[Mar 29 {\bf (***TUESDAY at 10:30*** --- CHL Conference Room (Building 3200, Room 200))}]High resolution modeling of urban and coastal flood inundation:
  data resources, unstructured meshing and resistance parameters. \\ Jochen Schubert (UC Irvine) \\
  Urbanized landscapes are at the forefront of current research
  efforts in the field of flood inundation modeling for two major
  reasons. First, urban areas hold a large amount of economic and
  social importance and as such it is imperative to protect them from
  flood related damage. Secondly, global changes of climatic patterns
  and sea level rise, combined with overdevelopment of watersheds and
  decay of ageing water retention and flood protection structures are
  increasing their risk of flooding. Flood inundation models are being
  developed to reliably forecast flow directions, velocities and
  depths of flood waters across a flooded site during a modeled flood
  event, with potential applications in urban planning and emergency
  response operations. However, modeling urban landscapes is not a
  trivial task since they consist of a multitude of geometrically
  complex features that are challenging to characterize within a
  computer modeling environment. Model parameterization has to capture
  site conditions such as preferential flow paths, drainage networks
  and surface dependent resistances to flow and often fine resolution
  geospatial data layers are required to capture enough detail to
  resolve said parameters at a meaningful level.

  This presentation will address advances in the design and
  parameterization of unstructured meshes for computationally
  efficient, fine resolution flood inundation modeling of urban
  environments at risk of flooding. Land-locked and coastal sites are
  presented along with necessary and available data to parameterize
  each scenario. Datasets for parameterization will range from
  offshore bathymetry to bay/river soundings, regional topography to
  local embankments and channels as well as buildings footprints and
  road network geometries. Findings will be presented based on the
  combined effects of mesh resolution, building and feature
  representation as well as landcover resistance parameterization on
  flood predictions for urbanized environments.

\item[Mar 23]USACE Reachback Experiences in Providing Support for International Hydrologic Emergencies\\ Mark Jourdan (CHL) \\
  The U.S. Army Corps of Engineers Reachback Operations Center (UROC)
  provides a ``reachback'' engineering capability that allows
  Department of Defense (DoD) personnel deployed worldwide to talk
  directly with experts in the United States when a problem in the
  field needs quick resolution. Deployed troops can be linked to
  subject matter experts within the Corps of Engineers, private
  industry, and academia to obtain detailed analysis of complex field
  problems. UROC staff members respond to incoming information
  requests such as flooding potential due to dam breaches; riverine
  flooding, river crossing operations, camp placement and other
  hydraulic/hydrologic evaluations.  This capability was initiated
  after support to the U.S. Army in crossing the Sava River in
  December 1995.  Since 1995, the UROC has grown to provide support
  fro not only hydraulic/hydrologic support, but also structural,
  geotechnical and environmental support to DoD personnel.  This
  presentation will discuss assorted hydrologic emergencies that UROC
  personnel have supported.  Lessons learned and approaches used will
  be discussed, as well as recommendations for future efforts in the
  discipline.

\item[Mar 16]Algorithms and Methods for Discrete Mesh Repair\\David McLaurin (MSU)\\
Computational analysis and design has become a fundamental part of product research, development, and manufacture in aerospace, automotive, and other industries.  In general the success of the specific application depends heavily on the accuracy and consistency of the computational model used.  The aim of this work is to reduce the time needed to prepare geometry for mesh generation.  This will be accomplished by developing tools that semi-automatically repair discrete data.  Providing a level of automation to the process of repairing large, complex problems in discrete data will significantly accelerate the grid generation process.  The developed algorithms are meant to offer semi-automated solution to a complicated geometrical problem — specifically discrete mesh intersection.
The intersection-repair strategy presented here focuses on repairing the intersection in-place as opposed to re-discretizing the intersecting geometries.  Combining robust, efficient methods of detecting intersections and then repairing intersecting geometries in-place produces a significant improvement over techniques used in current literature.  The result of this intersection process is a non-manifold, non-intersecting geometry that is free of duplicate and degenerate geometry.  Results are presented showing the accuracy and consistency of the intersection repair tool.

\item[Feb 7 {\bf (***MONDAY***)}]Numerical and Reduced Order Asymptotic Modeling of Nonlinear Deep-Water Ocean Surface Waves\\ Matt Malej (New Jersey Institute of Technology) \\ In this talk I will discuss the development of an accurate and efficient numerical model for a short-term prediction of evolving nonlinear ocean waves, including extreme waves such as ``Rogue'' waves. Due to their elusive nature, the media often portrays Rogue waves as unimaginatively huge and unpredictable monsters of the sea. To address these concerns we derived several asymptotic models based on the small wave steepness assumption, as well as relatively weak transverse dependence, and their corresponding numerical simulations via Fourier pseudo-spectral methods have been carried out. 
The derived models will be compared to the well-known Modified Nonlinear Schroedinger Equation (MNLS) of Trulsen and Dysthe (1996) for the evolution of the envelope of wave-trains. Motivated by their spectral evolution, the stability of a weakly nonlinear wave-train on deep water subject to three-dimensional modulations is investigated. Preliminary results on the validity of the MNLS and the computational advantages of our models will be discussed.

\item[Feb 2 {\bf (**Keulegen Room**)}]An outline of finite element methods for civil and environmental engineering, part 2\\Chris Kees (CHL)\\
The finite element method is a powerful approach for computing approximate solutions to partial differential equations, particularly when the equations are defined on complex domains and/or have highly variable coefficients. In the first talk of this series I presented some of the standard tools for dealing with partial differential equations including weak derivatives, the weak formulation, and Sobelev spaces. We finished with the abstract Ritz-Galerkin method for approximating solutions over a finite dimensional subspace of the infinite dimensional solution space. We saw that this solution has the Galerkin orthogonality and that the Ritz-Galerkin approximation is a kind of projection of the solution onto a finite dimensional subspace. In this talk I will show how to construct finite dimensional subspaces using the finite element method. We will cover some standard cell types and polynomial basis functions for constructing the solution representation on cells as well as the effect of constraints on inter-cell continuity.

\item[Jan 26] How to use the CMB to create and analyze 3D ADH
  groundwater and heat transport simulations\\Amanda Hines (ITL)\\ The
  Computational Model Builder is a suite of tools designed to help
  modelers create and initialize domains as well as analyze the
  results of their simulations. The CMB currently consists of 5 tools:
  PointsBuilder, SceneBuilder, ModelBuilder (which contains
  SimBuilder), MeshViewer, and ParaView. I will be presenting the
  current capabilities of the tools with respect to generating 3D ADH
  groundwater and heat transport domains. I will try to cover:
\begin{itemize}
\item Processing large scatter point datasets
\item TIN stitching
\item Well/veg/rock placements
\item Tagging of elements and nodes on large domains
\item Generating bc files for ADH (including material and boundary condition assignment)
\item Mesh quality
\item Reclassification of material IDs in a mesh
\item Analyzing simulation results in ParaView 
\end{itemize}
I will cover additional capabilities, such as the specialized 2D
tools, in a second presentation later this year.

\item[Jan 18 {\bf (***TUESDAY***)}] An outline of finite element methods for civil and
  environmental engineering, part 1\\ Chris Kees (CHL) \\ 
The finite element method is a powerful approach for computing
approximate solutions to partial differential equations, particularly
when the equations are defined on complex domains and/or have highly
variable coefficients. Since complex domains and heterogeneous
materials are the rule rather than the exception in civil and
environmental engineering, it makes sense for ERDC to invest in finite
element methods both through development of in-house expertise and in-house
software tools. In this first part of a series of informal talks on
the finite element method I will try to present the basic framework of
finite element methods for a second-order parabolic system of equations, which is
the type of partial differential equation derived for the large
majority of models in civil and environmental engineering. This first
part is intended as a refresher for those with some prior exposure
to finite elements and a background in real analysis and linear
algebra. Later parts will look at applications to problems across ERDC
and open issues in the development of finite element methods and
software.

\item[Jan 12] Open meeting on 2011 seminar topics

\end{enumerate}

\begin{center}
\Large
Year 2010
\end{center}

\begin{enumerate}

\item[Jan 27] Very large scale integrated hydrologic, hydraulic, and
groundwater modeling\\ Aaron Byrd (CHL) \\
There are many challenges to very large scale models (e.g.
river basin-sized) that span several states in size. The challenges
range from data collection and management, conceptual model creation,
numerical model creation, being able to run and calibrate the model, and
presenting results in a meaningful way to the stakeholders. Clearly
there are many significant changes that occur at this scale such as
urbanization and climate change and so being able to model large areas
would be beneficial for being able to manage them. This is intended to
be mostly a discussion of issues as well as how we might go about as a
group to tackle the problems and begin to model very large areas with
high levels of resolution. 

\end{enumerate}

\begin{center}
\Large
Year 2009
\end{center}

\begin{enumerate}

\item[Oct 28] Learning Whom to Trust: Using Graphical Models for
  Learning about
  Information Providers\\Philip Hendrix (ITL)\\
  In many multi-agent systems, information is distributed among
  potential providers that vary in their capability to report useful
  information and in the extent to which their reports may be biased.
  This talk describes a series of graphical models for learning about
  information providers in such settings. It shows how graphical
  models can be used to (1) simultaneously learn the reporting
  strategies that agents use and learn their capabilities; (2) weigh
  the benefits of different combinations of information providers; and
  (3) calculate the expected error of selecting different combinations
  of information providers. These models can cope with agents that
  vary in their capabilities and strategies, and whose capabilities
  may change over time.


\item[July 23] Morphology Modeling, Surface water/Groundwater,etc.\\Prof. Clint Dawson from the University of Texas at Austin\\
  A short informal seminar on some of his recent work, including modeling shallow water with morphological change using discontinuous Galerkin methods, surface water/groundwater coupling, and parameter estimation.

\item[March 27] ``Wiki For Dummies in GSL''\\ Victoria Moore\\
  As stated by ``Wikipedia'' a wiki is an online content management program designed to facilitation the collaborative editing of documents through and in-browser text interface, which are them immediately presented as publicly visible web pages.  The use of a wiki in GSL could promote information awareness through an unofficial means to which all can rapidly contribute.  The ``Wiki For Dummies in GSL'' brown bag seminar will cover an introduction to wiki's, adding/editing content, and managing data.  Please note that you need not be a dummy to attend.

\item[March 4] Discussion on local conservation in continuous finite elements\\ Charlie Berger\\
  Continuous FEM solutions of the conservative form differential equations will be shown to be identical to a finite volume method in which the fluxes at the edge are derived from the interior of each element.  The method then is locally conservative in that the ``mass'' change balances the sum of the fluxes.  The primary example will be of a second order partial differential equation, showing that the fluxes precisely balance the point source.

\item[February 25] A Krylov H2 optimal reduced order modeling technique applied to the two-dimensional linearized shallow water equations: A case study\\Chris Massey\\
  Over the past few years the US DoD's coastal ocean and riverine
dynamical system modeling capabilities
have taken large leaps forward. Direct numerical simulation using
modeling tools is a standard approach
and indeed one of the few available means to accurately predict
environmental features that are influenced
by riverine and near shore coastal processes. Yet, in the face of finite
computation resources and limited turn
around times, there is still a pressing need to reduce the cost of
evaluation of hydrodynamic models while at
the same time ensuring appropriate levels of their accuracy. In terms of
reduced order modeling (ROM),
hydrodynamic models can be seen as dynamical systems that involve a
description of how geophysical
flows evolve both in space and in time. ROM applied to these models is
very promising and has enormous
potential in the context of performing ensemble scenarios and data
assimilation where cost control and
sustainability of accuracy is critical. The Krylov-based projection
methods have emerged among the leading
approaches for model reduction in large-scale settings, producing high
quality models that satisfy (local)
optimality conditions, which is critical in retaining confidence in the
accuracy of the reduced system. The
linear framework for which this approach to model reduction is optimal
makes it even more appealing to
variational data assimilation approaches based on the representer
method.
In this talk, we present a case study in which we apply a Krylov H2
optimal reduced order modeling
technique, see [1], to the two-dimensional linearized shallow water
equations.\\References\\
 \ [1] S. Gugercin, A.C. Antoulas, and C. Beattie, "H2 Model reduction for
large-scale linear dynamical
systems," SIAM J. Matrix Anal. Appl., vol. 30, pp. 609-638, 2008.  \\ This is joint work with Christopher Beattie of Virginia Tech and Hans
Ngodock of the Naval Research Laboratory.

\item[February 11] Godfred Yamoah\\
  Grid coarsening schemes which ignore stored data at the deleted nodes when
computing for solutions at the next time step result in mass conservation
issues. We show the results of two mass conservation schemes which uses all
stored data (including data at the deleted nodes) and compare them with that
of the non-conservative method for which data of the deleted nodes are
ignored. The simulation is a 2 day, vertical infiltration problem. In
general, there is a loss of mass when the non-conservative method is used.

\end{enumerate}

\begin{center}
\Large
Year 2008
\end{center}

\begin{enumerate}

\item[Dec 31] Christmas Break

\item[Dec 24] Christmas Break

\item[Dec 17] Christmas Break

\item[Dec 11] Model formulations for air/water flow in porous media and related numerical modeling issues.\\ Chris Kees\\
  Richards' equation is the standard model for variably saturated water flow in the subsurface.  It is notoriously difficult to solve but still the primary continuum model used in hydrology.  ``Full'' two-phase models were developed for modeling oil and water in petroleum reservoirs.  Fast and accurate numerical models exist for certain two-phase flow formulations, but these models are still not widely used in hydrology for air/water flow.  I'll derive three standard model formulations that include compressibility and capillary effects and present some numerical test problems that Matthew Farthing and I have been working on.


\item[November 26] Thanksgiving Break

\item[November 19] Setting the Initial Condition and Boundary Conditions for the Countermine Heat Transport Problem\\ Owen Eslinger\\
  We will discuss and number of cases that I have run in which different values were set for the initial and boundary condition for the Countermine Problem.  All of these tests were performed on a column of soil with a single material.  The steady state solution was determined.  We will discuss the implications of setting this condition incorrectly for the surface heat flux.  We will discuss the ramp up time needed and the implications of not setting this correctly.

\item[November 12] Three-Dimensional Modeling of Problems in Biot Consolidation via a Mixed Least Square Finite Element Method\\ Dr. Parashkevova\\
  The Biot theory of consolidation, which models fluid-saturated porous elestic media by coupling the equations of elastic materials with those of Darcy flow, is central to the rational analysis of soil mechanics problems.  Application areas include geophysics (wave propagation), geotechnical engineering (soil-deformation problems), and industrial engineering (processing of fluid-solid mixtures).  Despite its importance, the application of the theory in practices is limited due to the difficulty in implementing the theory in numerical analysis codes using standard Galerkin finite element technology.\\
  The goal of this research is to develop a robust numerical method for Biot Consolidation in three dimensions using a mixed least-square method of approximation.  Previous work demonstrated the method for two dimensional problems.  The method prescribes separate unknowns for displacements and stresses for elasticity and pressures and velocities for Darcy flow.  The approximations for displacements and pressures can be either continuous or discontinuous, with no requirement for compatibility between approximation spaces for displacement/pressure and stress/veloctiy.  The method produces symmetric and positive definite coefficient matrices at variance to standard finite element approximation, which produce symmetric semi-definte matrices for incompressible materials.  The presentation presents mathematical formulations with numerical examples and outlines future developments.  The numerical examples suggest the method is convergent in both displacements/pressure and stresses/velocity.  Significantly, for elasticity the method works at nearly compressible and entirely incompressible limit of material behavior.

\item[October 29] Development tools in the HPC environment\\ Jeff Hensley\\
  A quick sample of some of the tools available on HPC platforms for debugging, code development, profiling, and optimization.  We will (briefly) debuggers (TotalView), memory debugging tools, profiling (CrayPAT and TAU) with an entertaining excursion into the IEEE standard for binary floating-point arithmetic and a look at the bizarre world of binary NaNs, infinities and assorted exotic creatures.

\item[October 22] Conceptual Model Builder: A Cross-Platform Framework of Tools for Model Creation and Setup\\
  Amanda Hines\\
  Numerous discipline-specific numerical models are researched, developed, supported, and maintained by the ERDC.  It is no longer feasible to provide mesh generation, boundary condition assignment, and even visualization capabilities on a single-processor desktop computer as the numerical models are being applied to larger and larger spatial domains.  In order to facilitate these large numerical simulations, the ERDC is developing a cross-platform framework of tools, the Conceptual Model Builder (CMB), for scene generations and initialization.  This talk will describe the requirements, the current status, and future development plans for the CMB.


\item[October 15] Isogeometric Analysis of Fluid Flow and Flow-Structure Interaction\\
  Yuri Bazilevs, University of California at San Diego\\
  Recently, Isogeometric Analysis has emerged as a new computational
  technology and as an alternative to the standard finite element
  method.  Isogeometric analysis improves upon finite elements in the
  areas of geometric modeling and solution representation. The first
  instantiation of isogeometric analysis was based upon non-Uniform
  Rational B-Splines (NURBS), although other alternatives, such as
  subdivision and T-Splines, are possible and are currently under
  investigation. Despite its recent emergence, NURBS-based
  isogeometric analysis has already been applied to many areas of
  contemporary interest in computational mechanics. These include:
  fluids and turbulence, solids and thin structures, fluid-structure
  interaction and, recently, phase-field modeling. Improved geometry
  and solution approximation properties of NURBS functions has led to
  superior performance of the isogeometric approach in comparison to
  standard finite elements in these applications. This presentation
  focuses on application of isogeometric analysis to wall-bounded
  turbulent flows, flows about rotating components, and vascular
  fluid-structure interaction. Basic ideas on how to develop discrete
  formulations that yield accurate and stable solutions for these
  applications are presented. Implementation of these ideas within a
  NURBS-based, large scale computational framework is discussed and
  computations that demonstrate good performance of the proposed
  methodology are shown.

\item[October 8] Introduction to multi-scale, multi-physics modeling with PyADH\\
Chris Kees\\
ERDC computational modeling research is now focused on
problems that involve many coupled physical processes which may
exhibit strong variation at a range of spatial and temporal
scales. Such multi-physics and multi-scale problems rarely have
analytical solutions and typically require sophisticated,
model-specific numerical methods, which may in practice be developed in
tandem with the mathematical models. Under these conditions, an
expert system for modeling is not feasible or desirable. Instead,
we need a flexible collection of tools to support development of
robust, accurate, and efficient computational models and methods.  In
this talk I will give an introduction to the design of PyADH, focusing
particularly on the model application interface and the model
development interface. To illustrate the interfaces I will give
examples of multi-scale, multi-physics models that we are developing
with the tools.

\item[October 1] Fall organizational meeting

\item[August 27] Data storage for computational mechanics\\
  Chris Kees\\
  Organizing data and implementing file readers and writers is not
  something I enjoy, but I end up doing it anyway.  I'll try to
  explain XML, HDF5, and XDMF, and then present some recent experience
  that we've had with these formats.  I intend to leave plenty of time
  for a discussion focused on long term approaches to data storage
  that we as a group could adopt.

\item[August 20] Plans for a Heat and Mass Transport Model of a Coupled Soil-Root-Stem System\\
  Jerry Ballard\\
  This brief presentation will describe the plans for the development
  of a three-dimensional tool that will simulate the heat and mass
  tranfer macro-interactions in a soil-root-system(SRSS) in a
  seasonally varying deciduous forest.  The development of the
  centimeter-scale SRSS will involve modifying and coupling heat and
  mass transport tools(such as ADH) capable of reproducing the
  three-dimensional diurnal internal and external temperatures,
  internal fluid distribution, and heat flow in the soil, roots, and
  tree trunks.  Technical challenges of the work include the
  calculation of the external thermal radiation from the surrounding
  forest into the SRSS, implementation of multiple thermal
  conductivity models, and the modeling of the active uptake of water
  in the soil by the root tissue.

\item[August 13] Two-Equation Turbulence Modeling\\
  Bob Bernard\\
  Turbulence is a predominant feature of real fluid flow, and some
  computational fluid dynamics (CFD) codes assume (by default) that
  the computed flow is always turbulent.  The range of length scales
  involved in turbulence is so great, however, that (with existing
  supercomputers) CFD codes can fully simulate turbulent flow only for
  Reynolds numbers of a few thousand or less.  For engineering
  purposes, the most common approach is to employ some kind of
  turbulence model.

\item[August 6] Reactive Transport (2): A pre-processor for biogeochemistry modeling\\
  Pearce Cheng\\
  A general biogeochemical system can be complex and difficult to
  resolve.  In this talk, we will first identify major difficulties of
  modeling biogeochemistry.  Then we will introduce a reaction-based
  pre-processor for biogeochemistry modeling.  An example will be used
  to show how this pre-processor can systematically decomposes
  reaction matrix and generates useful biogeochemistry governing
  equations.

\item[July 30]Ideas for Solving Large Problems with the Countermine Computational Testbed\\
  Stacy Howington and Owen Eslinger\\
  This brief talk will (1.)summarize the current state of simulations
  with the Countermine Computational Testbed (from the User's Group
  Conference presentation given two weeks ago), (2.) outline the
  simulation challenge given by our customers, (3.) list our current
  ideas for enlarging the domains substantially, and (4.) solicit
  ideas from the audience.

\item[July 9] Reactive Transport (1): Introduction\\
  Pearce Cheng \\
  What is reactive transport? We will discuss, biogeochemical
  reactions (conceptual models), reaction rate experiments, challenges
  in modeling reactive transport, and reaction rate issues.

\item[July 2] Mesh Manipulation and Air Flow Modeling\\
  Jeff Allen\\
  Mesh Manipulation Abstract: The need for realistic three-dimensional
  computer models of natural streams and rivers, particularly in light
  of the exorbitant costs associated with typical stream restoration
  has become a priority for many civil works researchers.
  Fortunately, the rapid advances and availability of high performance
  computing resources along with the increased sophistication of both
  in-house and commercial software have made the creation of such
  models significantly more tenable. In light of these needs, the
  objectives for this research include (1) the creation of realistic
  representations of naturally occurring streambeds from potentially
  coarse sets of field measurements; (2) the demonstrated capability
  to freely deform the streambed surface as well as embed complex
  objects within the surface (rocks, fallen trees, etc.); (3) the
  ability to successfully mesh the surface and its surrounding volume
  in accordance with established mesh quality criteria; and (4) obtain
  sufficiently resolved flow field solutions utilizing HPC resources.
  Beginning with a coarse set of field data measurements taken from
  one of four study sites along a 1.5-mile stretch along the Robinson
  Restoration project of the Merced River, California, for each
  objective stated, the respective challenges, solution strategies,
  and resulting outcomes are demonstrated. Flow field solutions are
  conducted using parallelized finite element/volume solvers.

  Air Flow Abstract: ADH is evaluated for its potential for solving
  incompressible, laminar air flow over several test case
  applications, including the flat plate at zero incidence, stationary
  cylinder, and the 10m x 10 m countermine terrain.  Comparisons with
  exact solutions, experimental, and the Fluent solver are shown.

\item[June 25] i-SciDat: An Ontology-Based Design for Scientific Data
  Storage \\
  Amanda Hines \\
  Over the past few years computer simulations have gained wider
  acceptance as viable alternatives to field testing. Due to this
  recognition, the numbers of simulations and types of simulations
  have increased significantly. This increase in computational
  modeling has produced the need for access to large amounts of data
  in a short amount of time. Also, it is not uncommon for data
  requirements to change at any moment, especially in a research
  environment such as the US Army Engineer Research and Development
  Center (ERDC). The ERDC recognized the need for a centralized,
  scientific data repository. It was essential for this repository to
  be easily accessible to users and flexible enough to handle the
  range of data required by the modelers.

  This presentation will discuss the repository, iSciDat, which has
  been created to address these issues at the ERDC. At the design
  stage and throughout the implementation of iSciDat, modelers were
  contacted to provide input about the process. Initially, these
  modelers were from the groundwater and surface-water fields. Based
  on their input, the repository's initial functionality was to store
  soil properties. After obtaining a list from the modelers, these
  properties were generated inside iSciDat. However, due to the
  evolving nature of research, the list has grown and will continue to
  grow. The best-equipped people to add or change these properties are
  the modelers themselves. This presented a dilemma since most
  modelers are not database experts. Therefore, the repository had to
  be created with enough flexibility that the modelers could easily
  manage the data definitions and relationships on their own. With
  this as the major driving force for design, we determined that an
  ontology-based repository was the best option. Ontologies provide
  the flexibility and hierarchical structure needed for this
  environment. This implementation of the repository allows modelers
  full control of their data storage without the daily need of a
  database manager and it also provides accessibility to store and
  retrieve data through the Internet. This presentation will discuss
  in depth the design rationale and implementation for the ERDC
  repository iSciDat.

\item[June 18]
Diverse Applications of the PAR3D Numerical Flow Model\\
Phu Luong and Bob Bernard

\item[June 13] Implicit Filtering \\
  Tim Kelley\\
  Many problems in optimal design are nonsmooth, nonconvex, and have
  internal iterations that may fail to converge. Traditional
  gradient-based optimization methods do poorly when confronted with
  these problems, often failing completely or finding an unacceptable
  local minimum. Implicit filtering is a sampling method. Sampling
  methods address these problems by only evaluating the function at
  each stage and then responding to the history of the optimization by
  targeting new points. Implicit filtering constructs a difference
  gradient, reduces the difference increment as the optimization
  progresses, and constructs a quadratic model with a quasi-Newton
  method. In this talk we will discuss the fundamental ideas of the
  method, the implementation, and an application to a problem in water
  resources policy.



\item[June 4] Optimization of Parameter Estimation in Groundwater Models Using Proper Orthogonal Decomposition (POD)\\
  Corey Winton\\
  We are solving inverse problems in groundwater
  modeling. Given values of hydraulic head at discrete locations, we
  seek to approximate values of hydraulic conductivity for the entire
  field.  When using ADH, extreme run-times prohibit the frequent
  function calls needed for parameter estimation for large, complex
  problems. Proper Orthogonal Decomposition (POD) is a method to
  reduce the size of the problem to calibrate ADH, reducing the number
  of full function calls needed. We will introduce the problem,
  discuss POD and how it is used, and demonstrate the accuracy of the
  POD solution compared to the full ADH solution.  If time permits, we
  will also discuss the current implementation of POD inside of ADH
  and where improvements could be made.  This is work done in
  conjunction with Tim Kelley, NCSU.

\item[May 28] Automated Mesh Generation for the Countermine Problem: Issues and Challenges\\
Owen Eslinger\\
As part of an ongoing effort to improve automated target recognition
algorithms for remote sensing technologies, a suite of closely coupled
numerical simulators has been developed. This platform includes
high-resolution thermal and moisture transport finite element models,
coupled with solar and vegetation models. It is well suited for the testing
of specific scenes with controlled environmental conditions, in particular
meteorological and time-of-day conditions, which otherwise might be
difficult and time consuming to reproduce in the field. However, this suite
serves as a complement to, rather than a replacement for, field and
laboratory testing of remote sensors.

These numerical simulations require the rapid and robust production of
finite element meshes of the shallow subsurface for the moisture and thermal
codes. The meshes need to include natural and man-made objects to achieve
realism and relevancy, e.g., rocks. This work will focus on the mesh
generation process, which has been accomplished by taking advantage of
open-source (black-box) mesh generation software and a post-process, mesh
smoothing technique to ensure quality elements in the final tetrahedral
mesh. In addition, the process allows for the inserted objects to be either
buried, flush with, or protruding from the ground surface. A simple mesh
repair operation around the objects is utilized to avoid poorly shaped
tetrahedra in these regions. When desired, subsurface soil regions can be
assigned as an additional post-process step. This ability extends to
statistically generated soil distributions using site-specific information
obtained from soil samples. The meshing software produced for this work has
been used and is available on a variety of HPC platforms.

\item[May 21] Building and Using a Macroscopic Network Model of Heterogeneous Porous Media\\
Stacy Howington 

\item[May 7] Testing ADH Unsaturated Flow Simulations Using Analytical
Solutions \\
Fred Tracy

\item[April 23] \htmladdnormallink{Mixed methods on the cheap (sometimes)}{talks/Farthing-4-23-08}\\
Matthew Farthing

\item[April 16] \htmladdnormallink{Some thoughts on the homogenization of granular media}{talks/Peters-4-16-08}\\
John Peters

\item[April 2 \& 9] \htmladdnormallink{Subversion, python, and other occult practices.}{talks/Kees-4-2-08}\\
Chris Kees

\item[Mar 26] \htmladdnormallink{Discrete element methods: the answer  to the granular media problem?}{talks/Peters-3-26-08}\\
  John Peters

\item[Mar 19] Conservative transport in heterogeneous porous media:  non-local theory and
network modeling.\\
Stacy Howington

\item[Mar 12] \htmladdnormallink{Modeling Time Dependent Viscoelastic Fluid Flow Using a Fractional Step
$\theta$-method.}{talks/Chrispell-3-12-08} \\
John Chrispell (Clemson University)\\
The accurate numerical approximation of viscoelastic fluid flow poses two
difficulties: the large number of unknowns in the approximating algebraic
system (corresponding to velocity, pressure, and stress), and the
different mathematical types of the modeling equations. An appealing
approximation approach is to use an operator splitting method which
decouples the conservation of momentum equation from the constitutive
equation. This split reduces the size of the linear systems that need to
be solved and separates the parabolic and hyperbolic equations into
different substeps. In this presentation, we describe the approximation 
method for the viscoelastic modeling equations, present numerical
simulations, and give a-priori error estimates.

\item[Mar 5] \htmladdnormallink{The trouble with granular media.}{talks/Peters-3-5-08}\\
  John Peters

\item[Feb 27] \htmladdnormallink{Calculating locally conservative velocity fields using discontinuous enrichment and other non-conforming finite elements.}{talks/Farthing-2-27-08}\\
  Matthew Farthing \\
  
  One serious deficiency of standard Galerkin finite element methods
  for flow modeling is that velocity fields obtained from
  differentiation of pressure fields are discontinuous at element
  boundaries. I'll present some recently developed local
  postprocessing schemes and a non-conforming finite element method
  that provide so-called locally conservative velocity fields. The
  resulting approaches maintain the basic flexibility and appeal of
  traditional finite element methods, while controlling nonphysical
  oscillations and producing element-wise mass-conservative velocity
  fields. 

\item[Feb 20] \htmladdnormallink{Conservative level set methods for incompressible air/water flow.}{talks/Kees-2-20-08}\\
  Chris Kees\\
  Standard level set methods do not conserve volume in incompressible
  multi-phase flows. The conservation errors are the result of
  describing the interface using the level set description and are not
  associated with any particular discretization. Since conservation
  errors accumulate over time to produce qualitatively incorrect
  simulations, this issue is very important. Several authors have
  tried to address it by using hybrid level set/volume of fluid and
  hybrid level set/particle tracking approaches. I'll present a method
  for correcting the level set to make it conserve volume or mass that
  is defined as the solution of a nonlinear variational problem and
  applies to general unstructured finite element methods.

\end{enumerate}
\end{document}
